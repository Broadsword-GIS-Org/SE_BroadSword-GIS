\documentclass{article}
\usepackage{hyperref}
\usepackage{graphicx}
\usepackage{float}
\hypersetup{
    colorlinks,
    citecolor=black,
    filecolor=black,
    linkcolor=black,
    urlcolor=black
}

\makeatletter
\renewcommand\paragraph{\@startsection{paragraph}{4}{\z@}%
            {-2.5ex\@plus -1ex \@minus -.25ex}%
            {1.25ex \@plus .25ex}%
            {\normalfont\normalsize\bfseries}}
\makeatother
\setcounter{secnumdepth}{4} % how many sectioning levels to assign numbers to
\setcounter{tocdepth}{4}    % how many sectioning levels to show in ToC
\begin{document}
		\begin{figure}[t]
			\centering
			\includegraphics[width=350px]{UP_Logo.PNG}
		\end{figure}
	
			\title{COS 301 Software Engineering: BroadSword-GIS (NavUP MODULE) DOCUMENTATION}
\maketitle
				
		\begin{flushright} \large
			Mankgwanyane Tlaka    - 14351872\\
			Johan	     du Plooy -	12070794\\
			Dimpho	     Mahoko   -	15175091\\
			Bernhard	 Schuld	  - 10297902\\
			Kamogelo	 Tsipa	  - 13010931\\
			Hendrik	van  der Mewe - 15101283\\
		\end{flushright}
		
		
		
		
		GitHub Repository: \href{https://github.com/Broadsword-GIS-Org/SE_BroadSword-GIS}\\
		\url{https://github.com/Broadsword-GIS-Org/SE_BroadSword-GIS}
	

\clearpage
\tableofcontents
\clearpage

\section{Scope}
Services to gather, maintain, persist and provide information related to fixed
spatial information needed for NavUP. It is about the creation and maintenance
 of a GIS Map of the campus and persisting information that can be
applied to determine the location of a device based on WiFi signal strengths
and other available sources of GIS information.
This module provide services to search for locations such as landmarks,
buildings, lecture halls, labs and any location added by an administrator.

\section{Use Cases}
	\subsection{Actors}
		\begin{itemize}
			\item Guest User: Get a location by it's name, ID or coordinates
			
			\item User: All Guest User functionality along with the ability to save frequent locations and link with timetable
			
 			\item Admin: All User functionality and CRUD any kind of singular element. 
			Import: Create a batch of elements via upload of a \textit{Json} File format
  			
				
		\end{itemize}
		
	\subsection{Use Case Diagram}
		\begin{figure}[H]
			\includegraphics[width=150mm]{GIS_Subsystem_Use_Case_Diagram_edit_by_kgothatso.png}
		\end{figure}
	
\section{Service Contracts}
	
		\begin{itemize}
 			\item \textbf{module.exports.get = function(req, res, next)} : Get all location elements in the system, return A Json  Array
  			\item  \textbf{module.exports.getByBuildingName} = function(req, res, next) : Get all locations in the given building, Return a Json Array
  			\item \textbf{module.exports.getRoute = function(req, res, next)} : Get the route between two points, return a Json array
  			\item \textbf{module.exports.create = function(req, res, next)} : Create a single location element, Persist in Database
  			\item \textbf{module.exports.delete = function(req, res)} : Delete a single location element from the database
  			\item \textbf{module.exports.patch = function(req, res, next)} : Update a single location element, persist changes in database
  			\item \textbf{module.exports.getById = function(req, res, next)}: Get a single location Element by ID, return Json object
			\item \textbf{module.exports.getBuildingNames = function(req, res, next)} : Get all buildings, return an array of building names
		\end{itemize}
\section{Tools}
	\subsection{Mean Stack}
		Every in this Module is build on top of the mean stack, in particular express. mean stack is chosen for its performance 
		. it allows for everything to be codded in JavaScript which saves time . Mean stack also support the MVC architecture.
	\subsection{Mongo DB}
		Mongo db allows for fast retrieval of large amount of data. Since it is a non-relational database it servers a good 				performance for NavUp GIS system.
	\subsection{Mongoose Schema}
		Everything in Mongoose starts with a Schema. Each schema maps to a MongoDB collection and defines the shape of the 		documents within that collection. 
		This works well since there is a need to exchange data in a standard format in between modules of NavUP.
	\subsection{Seed npm-packgae}
		This seed package is used to automatically populate the database when the server starts with location elements from a Json file format
	\subsection{NSQ}
		NSQ is a library that is used to communicate with other NavUP modules in real time
		
\section{Contribution}
	\begin{itemize}
		\item Issue Tracker: https://github.com/Broadsword-GIS-Org/SE_BroadSword-GIS/issues/
		\item Source Code : https://github.com/Broadsword-GIS-Org/SE_BroadSword-GIS/tree/master/GIS_SubSystem
	\end{itemize}
	
\section{Support}
	\begin{itemize}
		\item If you are having issues please let us know
		Mailing list: 
	\end{itemize}

\end{document}
